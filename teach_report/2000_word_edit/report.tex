\documentclass[a4paper,openany,nobib]{tufte-book}

%%math stuff
\usepackage{mathtools}
\usepackage{amssymb}

%%define colours
\usepackage{xcolor}
\definecolor{tred}{HTML}{450000}
\definecolor{tgray}{HTML}{161616}

%%pictures
\usepackage{graphicx}
\graphicspath{{./pics/}}
\setkeys{Gin}{width=\linewidth}%,totalheight=\textheight,keepaspectratio}

%%pagestyle
\usepackage[protrusion=true,expansion]{microtype}
\usepackage{booktabs}
\usepackage{fancyvrb}
\usepackage{lscape}
\fvset{fontsize=\normalsize}
\fancypagestyle{plain}{}

%%questionnaire macros
\usepackage{forloop}% used for \Qrating and \Qlines
\newcommand{\Qline}[1]{\noindent\rule{#1}{0.6pt}}
\newcounter{ql}
\newcommand{\Qlines}[1]{\forloop{ql}{0}{\value{ql}<#1}{\vskip0em\Qline{\linewidth}}}

%%references
\usepackage[backend=biber,citestyle=authoryear,bibstyle=numeric,firstinits=true,autocite=footnote]{biblatex}
\addbibresource{ref.bib}
\usepackage[nottoc,notlot,notlof]{tocbibind}

%%final styling
\title{Pupil misconceptions:\\ \noindent circuits and voltage}
\author{Zella Baig}
\publisher{\today}
%\color{tgray}
%\colorlet{darkgray}{tred}
\begin{document}
%%%%
%TODO: REMOVE 100 words
%%%%
\frontmatter
{\maketitle}
\tableofcontents
\thispagestyle{empty}
\mainmatter
\chapter{Note for Assessor - MSc in Social Data Science}
\setcounter{page}{1}
This document is a 1999 word extract from the final report I completed for the Teaching Physics in Schools short option offered by the MPhys Physics course in the second year. The report itself was $\sim$3000 words, and so I have cut out the 'Results' section, as well as the Appendix (which contained a copy of the questionnaire used, as well as outlines of the results themselves) and kept the rest of the document as I feel it represents my skills best for the MSc in Social Data Science. In particular, I believe the conclusion succintly goes over the results and also offers more insight into my views on the data I gathered.
\chapter{Background \& Outline}
It is well established that school-age children often struggle with the concept of {'electricity'\autocite{psillos}}.
This manifests in many ways;
via misconceptions regarding physical processes which occur in circuits, their respective mechanisms, or through manipulations of circuitry. Much discussion has gone into examining the origins of these misconceptions,
which lead to students conceptualising 'electricity' incorrectly.
Regardless, it is known that these misconceptions are both difficult to identify, and unlearn (despite targeted education on these {topics)
\autocite{lee2001}}.
The aim of this report is to examine these misconceptions with both primary data collected from a local school, as well as secondary reading.

One concept which appears repeatedly as causing difficulty is voltage: not only is this concept fundamentally {misunderstood \autocite{shipstone_children}}, but when it \emph{does} play a part within students' reasoning for circuit theory it often plays a secondary role to other concepts (chiefly, current).
This may, perhaps, be related to the method of teaching. 
Härtel discusses how students' reasoning seems to follow the {order\autocite{hartel82}}:
\begin{align*}
	\text{Current}\rightarrow \text{Charge} \rightarrow \text{Voltage} \rightarrow \text{Resistance}
\end{align*}
which is to say that voltage is only considered \emph{after} students have examined the current and charge responses. This is problematic given how \emph{voltage} is often the initial concept \emph{causing} the various responses in the current.

Before discussing the misconceptions which students hold when analysing circuits, it is worth examining the reasons \emph{why} they might hold the ideas they do.
Relevant to the previous point made about discussion of energy, many of these texts (such as Härtel, 1982) discuss how students refuse to think in terms of energy in circuits (or often get confused with 'current' or 'electricity').
Another point is the idea of models.

Upon learning new ideas students are likely to adopt various models to help explain these {ideas\autocite{bagno2000}}, but often encounter difficulty within circuits \& electricity to do so due to a lack of real-world links between the microscopic behaviour within the circuits to macroscopic phenomena. Students have incorrect models, misapply models, or are unable to map behaviours between models and 'actual' circuit phenomena.

As Gutwill et al. state, there is difficulty in linking ``mechanisms'' \& {``representations''\autocite{gutwill99}}.
Students usually view circuits from three different perspectives:
\begin{enumerate}
	\item Microscopic, e.g. charge carriers
	\item Aggregate, e.g. considering current and potential difference
	\item Topological, e.g. open/closed circuits and distances between components
\end{enumerate}
Bearing in mind the previous discussion (students having ill-defined ideas of various circuit phenomena outright), one may see how students would encounter difficulty in linking their physical intuitions not only to each of the perspectives individually, but also of trying to relate changes in one to those in another, given how sub-concepts are already poorly defined. More generally, it appears that students think of perturbations of circuits in three broad {models\autocite{ates2005}}:
\begin{enumerate}
	\item Sequentially, where current is affected by changes as it travels
	\item Locally, where perturbations are contained within a branch of the circuit
	\item Via superposition, where changes can be seen to be 'stacked' on top of pre-existing conditions
\end{enumerate}
When looking at the exact ``mechanisms'' employed within these models, it is worth discussing \emph{phenomenological primitives}, or \emph{p-prims} for {short\autocite{prim}}. 

P-prims (representing intuitive ideas which ``are usually evident in our everyday experience''), or their lack, may be one avenue through which these misconceptions arise. If the knowledge which students are fed is obfuscated from their own lines of reasoning - either from superfluous teaching of models when ideas may be self developed or being taught in too abstract of a manner to pick up p-prims, students may not only pick up incorrect ideas but find it more difficult to adjust their misconceptions. 
This conclusion is backed up by Ugur et {al.\autocite{ugur}}, where it was found that misconceptions remained despite targeted teaching.

Expanding upon this, in work done by Chi \& Slotta\autocite{slotta} there is evidence to suggest that the difficulties in constructing correct models and changing incorrect ones, lie with the ontological categorisation of the ideas involved: Lee and Law\autocite{lee2001} discuss how:
\begin{itemize}
	\item Students categorise ideas as either 'matter' based, or 'process' based;
	\item Students seem to prefer 'matter' based models of circuits;
	\item Evidence suggests that 'process' based models provide better understanding of circuit behaviour;
	\item And perhaps most interestingly, that voltage was thought of as 'process' based more often than other circuit concepts.
\end{itemize}
What we see here links to our earlier discussion on p-prims:
in seeing concepts as 'matter' based perhaps students are employing intuitive matter-based p-prims (and extending this, may lack the necessary p-prims to conceptualise voltage in the same manner). This, along with the preference to categorise voltage as a 'process' concept may be evidence that students have a lack of intuitive parallels for voltage than for other concepts in circuit theory - and that teaching might need to specifically address this.

Examining the models themselves, Osborne\autocite{osb} identified several recurring misconceptions which students seemed to have, and in fact still seem to {have in modern education\autocite{suryadi2020}}:
\begin{itemize}
	\item The battery as a source of \emph{current} 
	\item Current being 'used up' by components as it travels
	\item \emph{Sequential reasoning}, where changes propagate along the flow of current
	\item \emph{Local reasoning}, where only isolated parts (e.g. a single branch) of the circuit are examined
\end{itemize}
Again, several of these misconceptions can be tied directly to a lack of conceptual understanding of the 'changes' which occur within a circuit - which itself links to the notion of voltage and what that represents in terms of energy flow; similar views have been expressed in literature such as Eylon and {Ganiel\autocite{eylon1990}}.

Thus, this report shall focus on examining the misconceptions which students hold with the inter-relations of energy, voltage, \& current, and seek to examine the conceptual processes which students undergo when dealing with these ideas.
\newpage
\chapter{Methodology}%
The design of the surveys given to students was largely based off of those conducted in previous literature, such as by Shipstone et al., Afra et al., or Küçüközer and Kocakülah\autocite{shipstone_europe,afra2009,kucu2007}. In essence, the survey has been designed to incorporate a range of short and long-form responses, based upon circuit diagrams which are given to the students. When used, all numerical values were chosen to be integers such that the focus of the student remain on the problem conceptually, rather than numerically.

The survey was given to two Year 9 classes and three Year 10 classes, for a total of 84 responses (split $35+49$ for each year group respectively). The Y9 group had covered electricity at a pre-GCSE level, and the Y10 group at a GCSE level. The survey was split into 3 sections:
\begin{itemize}
	\item[S1.] Series and Parallel Circuits
	\item[S2.] Circuit Ideas
	\item[S3.] Circuit Reasoning
\end{itemize}

S1 dealt with steady-state behaviour and changes in both series and parallel circuits with multiple resistances. With the exception of a question asking for an explanation of any perceived differences between series and parallel circuits, all the prompts were multiple choice. This section was designed to be the most standard - akin to GCSE style questions. The aim of this section was to identify any glaring misunderstandings of circuit behaviour and perturbations.

S2 broadly covered two topics:
that of potential difference (examining what it was thought to be and what purpose it served), and that of differences \& perturbations within circuits at a conceptual level; that is to say examining how students approach these concepts in relation to ideas such as ``charge'' or ``potential difference''. 
To this end, no numerical responses were required (though basic component values were given for clarity).

S3, the final section, was purely open ended and asked the students to compare a series and parallel configuration of bulbs in an otherwise identical configuration, and also to highlight any further difficulties with the idea of ``voltage''.
The first question in this section was deliberately left open-ended, as to be able to ascertain what links and comparisons with bulb brightness to circuit ideas could be drawn up by the students themselves - in short, this section served to draw out their self-employed methods for tackling circuit problems.
\chapter{Conclusion}%

The results regarding students' misconceptions with the nature of voltage are directly in line with what literature suggests - namely that it is ill-understood. The results of S1 \& S2 also demonstrated that students' internal models may be flawed - with a lack of focus on energy changes being the ultimate cause of circuit phenomena, as discussed in work by Eylon and Ganiel. We see how this may credit the work done by Chi \& Slotta, demonstrating a 'matter' based approach to circuit reasoning, although this has not been verified within this study.

We also see evidence of the models proposed by Osborne, such as the current being used up and perturbations occurring sequentially. Furthermore, we also see evidence in some scenarios (such as in Q12) that further education was able to dispel these misconceptions.

The key concerns brought up by this study seem to be:
\begin{itemize}
	\item A misunderstanding of Ohm's law
	\item Unclear links to energy within circuits
	\item Possible rote memorisation of tools rather than understanding the logic behind them
\end{itemize}
with the first point appearing repeatedly in S1 \& S2, where a higher resistance was linked to both a higher current and higher voltage across the components in question.
The second point is brought up more perhaps in the extended responses, in justifications for decisions made in earlier questions. Here we see flaws in the internal model(s) used by students; they rarely seem to incorporate the transfer and transformation of energy, and instead rely on a 'dynamic' model - again demonstrating what can be thought of as matter-based p-prims.
\marginnote{This dynamical preference illustrated well, for example, by one student who stated that an ammeter would \emph{``notice the change of resistance before [the other ammeter]''}}
Lastly, an issue which might be difficult to pick up in the classroom is the third point raised: the justifications given for reasoning often seem to imply that students are parroting facts about what circuits behave like,
as opposed to understanding why it is that a given phenomena occurs, best demonstrated in the questions where students were asked to justify reasoning: many simply stated that circuits were series/parallel.

These areas as well as the other well-established misconceptions regarding voltage should be both investigated and targeted within the classroom. It appears that there is a gap in the existing models with regards to both the inclusion of energy to an appropriate degree, as well as a lack of focus on voltage, which leads to poor understanding and incorporation of these concepts within reasoning. Further, it appears that Ohm's law is one tool where students perform poorly (being unable to manipulate the equation conceptually, without values) and that this may tie into the previous point: Ohm's law may be fundamentally misunderstood or difficult to conceptualise.
Importantly, the current form of education must seek to move away from encouraging memorisation of facts and instead towards understanding why they hold; energy would naturally appear here more frequently given its fundamental role in much of physics and so perhaps a shift in teaching away from focusing on the \emph{dynamical} aspects of models to the \emph{energy} aspects is required as to instill these concepts fully. For example, in teaching circuits using a waterfall model, linking more strongly the GPE water has at the top to the energy given by the battery, and explaining the splitting of voltage/current in series/parallel through the usage of turbines in the water stream arranged in series/parallel alongside potential energy changes. Care must be taken not to link these ideas too strongly to everyday dynamical models (or rather, to make it clear how said models can be overextended).
Further, this does present the difficulty in forming links to well-established phenomena in students' lives: further investigation is needed as to the exact nature of the links between circuits and everyday occurrences.

Ultimately, while the study was performed only at one school using a single method of teaching, there was clear improvement over the 2 year groups examined; this is a positive outcome as it suggests that further targeted education on these specific areas may seek to reduce these misconceptions further; with further study we should be able to narrow down the sources of misconceptions and help prevent them from forming.
\backmatter
\nocite{*}
\printbibliography[heading=bibintoc]
\end{document}
